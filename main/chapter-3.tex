\chapter{Spiralmodellen}
\label{chapter:3rdchapter}
Som primært opslagsværk er artiklen \cite{boehm1988spiral} blevet anvendt. 

\textbf{Baggrund:}\\
Spiralmodellen blev introduceret af Barry Boehm i 1988 i hans artikel "A Spiral Model of Software Development and Enhancement." - som også er anvendt som primær kilde. 
Boehm udviklede Spiralmodellen som et forsøg på at kombinere styrkerne fra både vandfaldsmodellen og den iterative udviklingsmodel, 
samtidig med at han adresserede de mangler, som traditionel softwareudvikling havde, når det kom til risikohåndtering.

Spiralmodellen blev populær i store og komplekse softwareprojekter, især i miljøer med høj risiko, 
såsom forsvars- og rumfartsindustrien. Modellen tilbyder en ramme, hvor risikostyring er centralt for hver udviklingsfase, 
hvilket hjælper teams med at navigere igennem komplekse problemstillinger og reducere sandsynligheden for dyre fejl senere i processen.
Det fornemmes klart på modellen, at det er en meget pseudovidenskabelig tilgang til softwareudvikling. 
Blandet andet anvendes begreber som det angulære moment i spiralen, som et udtryk for fremdrift - Hvis et burndown chart er en DJØF'ers våde drøm, så må det være ingeniørens ditto.

\textbf{Formål og fokus:}\\
Spiralmodellen har sit fokus på risikostyring. Hver iteration, eller "spiral," begynder med en analyse af risici, 
hvorefter der træffes beslutninger om, hvordan udviklingsarbejdet skal fortsætte. Modellen giver en fleksibilitet, 
som gør det muligt at gå tilbage og revurdere tidligere beslutninger, hvilket gør den ideel til projekter, hvor kravene kan ændre sig eller er ukendte fra starten.

\textbf{Fremdrift og styring af fremdrift:}\\
I Spiralmodellen er fremdriften organiseret i faser, hvor hver iteration repræsenterer en udviklingscyklus, der slutter med en leverance. 
Hver cyklus starter med en planlægningsfase, hvor risici vurderes, og projektet evalueres, før det bevæger sig videre til næste iteration.

\textbf{Kvalitet i produktionen:}\\
Kvaliteten sikres gennem den iterative proces, hvor produktet konstant vurderes og forbedres i hver spiral. 
Den regelmæssige evaluering af både risici og funktionalitet medvirker til, at man undgår store fejl, og at der er løbende fokus på kvalitet.

\textbf{Dokumentation:}\\
Spiralmodellen lægger vægt på grundig dokumentation i hver fase af projektet. 
Dette sikrer, at både risikovurderinger, tekniske beslutninger og projektets fremgang er veldokumenteret, hvilket gør modellen velegnet til projekter med krav om formel dokumentation.

\textbf{Sikring af, at "det rigtige bliver lavet rigtigt":}\\
Spiralmodellen sikrer, at det rigtige produkt udvikles gennem løbende evaluering af både krav og risici i hver iteration. 
Ved afslutningen af hver cyklus revurderes projektet, og der tages beslutninger om, hvordan udviklingen skal fortsætte.

\textbf{Vidensdeling og teamsamarbejde:}\\
Spiralmodellen har mindre fokus på daglig vidensdeling sammenlignet med XP, men sikrer, 
at alle teammedlemmer er opdaterede gennem den formaliserede planlægnings- og dokumentationsproces, der findes i hver fase.

\textbf{Risikostyring:}\\
Risikostyring er et centralt element i Spiralmodellen. Hver iteration starter med en grundig risikovurdering, 
som styrer beslutningerne for den kommende fase. Det gør modellen særligt velegnet til komplekse projekter med stor usikkerhed.

\textbf{Budget og tidsstyring:}\\
Spiralmodellen håndterer budget og tidsstyring gennem detaljeret planlægning i hver fase. 
Den formaliserede struktur med klare milepæle og risikostyring gør det lettere at kontrollere budgettet og overholde tidsplanen sammenlignet med mere fleksible metoder som XP.