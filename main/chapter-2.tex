\chapter{XP}
\label{chapter:2ndchapter}
Som primært opslagsværk er bogen \cite{beck2004xp} blevet anvendt.

\textbf{Baggrund:}\\
Extreme Programming (XP) blev udviklet af Kent Beck i slutningen af 1990'erne. 
Det blev oprindeligt introduceret som en måde at forbedre softwareudviklingsprocessen i situationer, 
hvor kundens krav ofte ændrede sig. XP blev født som en reaktion på de udfordringer, 
traditionelle vandfaldsmodeller stod over for, når de skulle håndtere skiftende krav og behov for fleksibilitet.

Kent Beck begyndte at eksperimentere med XP, mens han arbejdede på et projekt for Chrysler Corporation. 
Han fokuserede på at fremme hurtig feedback, hyppig udgivelse af små funktionaliteter og tæt samarbejde mellem udviklere og kunder. 
Beck offentliggjorde først sine ideer om XP i 1999 med bogen "Extreme Programming Explained: Embrace Change." 
Modellen fik hurtigt momentum blandt agile tilhængere og blev en af de mest populære agile udviklingsmetoder.
Som en sidenote, så blev Chrysler-projektet aldrig færdiggjort - much in the spirit of Agile.. 

\textbf{Formål og fokus:}\\
XP fokuserer på at levere høj kvalitet og værdifuld software gennem hyppige iterationer, hvor koden konstant forbedres. 
Den understøtter praksisser som parprogrammering, test-drevet udvikling (TDD) og kontinuerlig integration, hvilket hjælper teams med at levere funktionalitet hurtigere og med færre fejl.
XP en af de mest ekstreme agile metoder, og den er ikke nødvendigvis egnet til alle projekter.

\textbf{Fremdrift og styring af fremdrift:}\\
XP styrer fremdriften gennem korte iterationer (cyklusser) på 1-2 uger. I hver cyklus leveres ny funktionalitet, og udviklerne samarbejder tæt med kunden. Projektet tilpasses løbende, så ændrede krav hurtigt kan implementeres, hvilket (burde) sikrer konstant fremdrift.

\textbf{Kvalitet i produktionen:}\\
Kvalitet sikres gennem teknikker som parprogrammering, test-drevet udvikling (TDD) og hyppige kodegennemgange. XP fokuserer på at opfange fejl tidligt, hvilket skulle kontinuerligt forbedrer kodekvaliteten.

\textbf{Dokumentation:}\\
Dokumentationen i XP er minimal. XP lægger vægt på at skabe forståelse gennem koden og kommunikationen i teamet, frem for at nedfælde alting skriftligt. Kun det mest nødvendige dokumenteres.

\textbf{Sikring af, at "det rigtige bliver lavet rigtigt":}\\
XP sikrer, at det rigtige produkt bliver udviklet gennem tæt samarbejde med kunden. Kunden leverer feedback i slutningen af hver iteration, så produktet konstant kan tilpasses efter deres behov.

\textbf{Vidensdeling og teamsamarbejde:}\\
Vidensdeling i XP sker gennem daglige stand-up møder og tæt samarbejde mellem teammedlemmer. Parprogrammering sikrer, at viden deles mellem udviklerne, og at ingen sidder alene med kritisk viden.

\textbf{Risikostyring:}\\
XP håndterer risici ved at bryde arbejdet op i små iterationer, så ændringer kan imødekommes hurtigt. På grund af den korte feedback-cyklus kan problemer løses tidligt, men risikostyring er ikke formelt struktureret som i andre metoder.

\textbf{Budget og tidsstyring:}\\
XP prioriterer hurtig levering af funktionalitet, hvilket kan give god kontrol over budgettet, da man kan justere omfanget af funktioner undervejs. Den fleksible tilgang kan dog skabe risiko for budgetoverskridelser, hvis omfanget ikke kontrolleres.