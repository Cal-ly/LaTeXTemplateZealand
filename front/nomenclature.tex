\chapter*{Nomenclature}
\addcontentsline{toc}{chapter}{Nomenclature}

\section{Abbreviations and definitions}
\label{sec:abbreviations}
The following abbreviations and definitions are used throughout the report:

\renewcommand{\arraystretch}{2}
\begin{longtable}{l p{13.5cm}}
    \textbf{Abb.}  & \textbf{Definition}  \\ \hline
        DB           & Database: A structured collection of data stored electronically. \\ \hline
        RDB          & Relational Database: A type of DB that stores and provides access to data points that are related to one another. \\ \hline
        DBMS         & Database Management System: Software that handles the storage, retrieval, and updating of data in a database. \\ \hline
        RDBMS        & Relational DBMS: A database management system based on the relational model introduced by E.F. Codd. \\ \hline
        SQL          & Structured Query Language: A programming language used to manage and manipulate relational databases. \\ \hline
        CRUD         & Create, Read, Update, Delete The four basic operations of persistent storage: adding, retrieving, modifying, and removing data. \\ \hline
        GUI          & Graphical User Interface  A user interface that allows users to interact with electronic devices using graphical icons and visual indicators. \\ \hline
        PK           & Primary Key:A unique identifier for each record in a database table. \\ \hline
        FK           & Foreign Key: A field in a database table that links to the primary key of another table. \\ \hline
        NML          & Normalization: In relation to RDB design, the process of organizing the columns (attributes) and tables (relations) to minimize data redundancy. \\ \hline
        1NF          & First Normal Form: A stage of NML where each table has atomic (i.e. indivisible) values and each record needs to be unique. \\ \hline
        2NF          & Second Normal Form: A stage of NML where it meets all the requirements of the 1NF and does not have partial dependency. \\ \hline
        3NF          & Third Normal Form: A stage of NML where it meets all the requirements of the 2NF and has no transitive functional dependencies. \\ \hline
        ERD          & Entity Relationship Diagram: A graphical representation of entities and their relationships to each other, typically used in database design. \\ \hline
        UML          & Unified Modeling Language: A standardized modeling language used to specify, visualize, construct, and document the artifacts of software systems. \\ \hline
        DDL          & Data Definition Language: A subset of SQL used to define database structures, such as tables, schemas, and databases. \\ \hline
        DML          & Data Manipulation Language: A subset of SQL used for adding (inserting), deleting, and modifying (updating) data in a database. \\ \hline
        DCL          & Data Control Language: A subset of SQL used to control access to data in a database. \\ \hline
        TCL          & Transaction Control Language: A subset of SQL used to manage the changes made by DML statements. \\ \hline
        XML          & Extensible Markup Language: A markup language that defines a set of rules for encoding documents in a format that is both human-readable and machine-readable. \\ \hline
        JSON         & JavaScript Object Notation: A lightweight data-interchange format that is easy for humans to read and write, and for machines to parse and generate. \\ \hline
    \renewcommand{\arraystretch}{1}
\end{longtable}